\documentclass{article}
\usepackage[utf8]{inputenc}
\usepackage[a4paper, margin=1in]{geometry}
\usepackage{natbib}
\usepackage{graphicx}
\usepackage{amsmath,amsthm,amssymb}
\usepackage{parskip}
\usepackage{hyperref}
\hypersetup{
    colorlinks=true,
    linkcolor=blue
}

\setlength\parindent{0pt}

\begin{document}
\section*{Heat Equation With Forward Euler}
We will solve heat equation 
\begin{equation} \label{eq:heat_eq}
u^{\prime} = \alpha \Delta u
\end{equation}
using forward Euler. 

We use the discrete Laplacian operator for triangle meshes:
\begin{equation} \label{eq:discrete_laplacian_op}
\Delta u_i = \frac{1}{2}\sum_{j} (\cot{\alpha_{ij}} + \cot{\beta_{ij}})(u_j - u_i)
\end{equation}

Applying forward Euler to Eq(\ref{eq:heat_eq}), we get for each vertex $i$:
\begin{equation} \label{eq:update_rule}
u_i^{k+1} = u_i^{k} + \tau \alpha \Delta u_i^{k}
\end{equation}

%Substituting Eq(\ref{eq:discrete_laplacian_op}) into Eq(\ref{eq:update_rule}), we have
%
%\begin{equation} \label{eq:update_final}
%F(u_i^{k+1}) = -\frac{\tau\alpha}{2} \sum_{j}(\cot \alpha_{ij} + \cot \beta_{ij}) u_j^{k+1} + \left[\frac{\tau\alpha}{2}\sum_{j}(\cot \alpha_{ij} + \cot \beta_{ij}) + 1\right]u_i^{k+1}  - u_i^k = 0
%\end{equation}
%
%We are going to use Newton Method to solve Eq(\ref{eq:update_final}) for $u_i^{k+1}$.
%And According to Newton Method, we can numerically solve Eq({\ref{eq:update_final}}) by the following iteration
%\begin{equation} \label{eq:newton_iter}
%	u_i^{{k+1}[n+1]} = 	u_i^{{k+1}[n]} - \frac{F(u_i^{{k+1}[n]})}{F'(u_i^{{k+1}[n]})}
%\end{equation}

%Eq(\ref{eq:update_final}) is a linear system, which can be assembled into matrix form and passed into a linear system solver. We assemble all unknowns in the following form:
%\begin{equation}\label{eq:unknowns}
%x = \begin{bmatrix}
%       u_0^{k+1} \\[0.3em]
%       \vdots \\[0.3em]
%       u_{n-1}^{k+1}
%     \end{bmatrix}
%\end{equation}
%where $n$ is the number of vertices in the mesh. The coefficients can be assembled into an $n$-by-$n$ matrix $A$:
%\begin{equation} \label{eq:laplace_mass_matrix}
%A(i, p) =
%  \begin{cases}
%    \frac{\tau\alpha}{2}(\cot \alpha_{ip} + \cot \beta_{ip})     & \quad \text{if } p \text{ is a neighbor of } i\\
%   -\frac{\tau\alpha}{2}\sum_{j}(\cot \alpha_{ij} + \cot \beta_{ij}) - 1  & \quad \text{if } p = i\\
%   0 \quad \text{ otherwise}
%  \end{cases}
%\end{equation}
%
%Since a vertex will usually have a few neighbors in a mesh, $A$ will be a sparse matrix.
%
%We also need $b$ on the right side of the linear system. $b$ is constructed by extracting the right hand side of Eq(\ref{eq:update_final}) and put them into corresponding positions:
%\begin{equation} \label{eq:rhs}
%b = \begin{bmatrix}
%       -u_0^{k} \\[0.3em]
%       \vdots \\[0.3em]
%       -u_{n-1}^{k}
%     \end{bmatrix}
%\end{equation}
%
%After everything is assembled, we only need to solve $Ax = b$ for $x$ using \href{http://eigen.tuxfamily.org/index.php?title=Main_Page}{Eigen} package.

\end{document}

